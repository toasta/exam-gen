\documentclass{extarticle}

\usepackage{scrlayer-scrpage}
\usepackage{geometry}
\usepackage{adjustbox}
\usepackage{wrapfig}
\usepackage{longtable}
\usepackage{calc}
\usepackage{multirow}
\usepackage{multicol}
\usepackage{tabularx}
\geometry{
a4paper,
left=20mm,
right=20mm,
top=20mm,
bottom=20mm,
}
\usepackage{graphicx}
\usepackage{amsfonts}
\usepackage{amssymb}
\parindent0mm
\setlength{\footheight}{14pt}
\setlength{\headheight}{14pt}



%\begin{minipage}{ {{common.marker_width}} }
\adjustbox{valign=m,raise=-.1em}{
    \includegraphics[width={{common.marker_width}}]{common/out/{{num}}.png}
    }
%\end{minipage}


\ihead{ {{ marker(0) }} }
\ohead{ {{ marker(0) }} }
\ifoot{ {{ marker(0) }} }
\ofoot{ {{ marker(0) }} }
\cfoot{\pagemark}


\begin{document}
\pagestyle{scrheadings}





\iffalse
{{ sheet.json_readable }}
\fi


\begin{wrapfigure}{r}{ {{common.main_qr_width}} }
\includegraphics[width={{common.main_qr_width}}]{{ '{'}}{{ sheet.qr }}}
\end{wrapfigure}

    { \large {{ sheet.name}}} Unterschrift \fbox{\hspace{50mm}\vspace{20mm}}

%\begin{multicols}{2}[Hinweise: ]
    \begin{itemize}
    \item Sie koennen die Felder wie es ihnen beliebt markieren: Haeckchen, Kreuze, Kreise, ausmalen.
    \item alle Fragen koennen mehrere oder auch keine richtige Antwort haben.
    
    \item Sollten Sie einen Fehler korrigieren wollen:
    \begin{itemize}
    \item Markieren sie, dass ihre Loesung in der naechsten Spalte steht, indem sie das Kaestchen der Spalte markiern.
    \item Markieren sie in dieser neuen Spalte \textbf{alle} richtigen Loesungen.
    \item Als von Ihnen eingerichte Loesung gilt immer die am weitesten rechts liegende von Ihnen markierte Spalte.
    \end{itemize}
    \item Die Barcodes und Marker duerfen nicht veraendert werden.
    \item Den Bereich rechts der Marker jeder Frage duerfen Sie beschreiben.
    \item Vergessen Sie nicht, den Test zu unterschreiben.
    \item Antwortmoeglichkeiten sind nicht auf mehrere Seiten aufgeteilt.
    \item Beachten sie eine moegliche Rueckseite.
\end{itemize}

Viel Erfolg!

%\end{multicols}




\begin{longtable}[l]{r |  c c c | l }

    
        & {{ marker(1) }}
    
    \endhead
    
    
        
        \\*
        
        \multicolumn{5}{l}{
            ({{i.points}}P) \textbf{ {{ i.q }}}
        }
        \\*
        \\*
        {{ marker(2) }}
            & $\boxtimes$

                
                        & $\square$
                

                    & {{ marker(4) }}
                    \\*

                
                %\noindent\parbox[c]{\hsize}{ {{ j }} }
                {{ j }} 

                    
                        & $\square$
                    

                    &  {{ marker(4) }} 

                     
                        {{ marker(3) }} \\
                     
                    \\*
                     

                
    
    \end{longtable}


    \cleardoublepage
    


\end{document}
